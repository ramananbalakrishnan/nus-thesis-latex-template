\documentclass{nusthesis}

% --------------------------------------------------
% Basic information
% --------------------------------------------------
\title{Quad Polarization Wideband Sinuous Antenna Elements and Arrays}
\author{Ramanan Balakrishnan}
\qualification{B.Eng. (Hons.), NUS}

\degree{Master of Engineering}
\university{National University of Singapore}
\department{Department of Electrical and \\Computer Engineering}
\submityear{2014}

\date{\today}

\begin{document}

% --------------------------------------------------
% Build the cover page, title page, declaration,
% dedication and acknowledgement pages
% --------------------------------------------------
\frontmatter
\maketitle
\declarationpage{}

\newpage
\thispagestyle{empty}
\begin{flalign*}
&&\nabla \cdot \mathbf{E} &= \frac{\rho}{\epsilon_0 \nonumber}\\
&&\nabla \cdot \mathbf{B} &= 0 \nonumber \\
&&\nabla \times \mathbf{E} &= - \frac{\partial B}{\partial t} \nonumber \\
&&\nabla \times \mathbf{B} &= \mu_{0}\mathbf{J} + \mu_{0}\epsilon_{0}\frac{\partial E}{\partial t} \nonumber
\end{flalign*}
\hfill \emph{and there was light}

\newpage
\acknowledgment{
Let's thank some people here.
}

% --------------------------------------------------
% Table of contents, abstract,
% lists of tables, figures and symbols
% --------------------------------------------------
\tableofcontents
\newpage
\abstract{
A section to summarize the main contributions of this thesis.
}

\listoftables
\listoffigures
\listofsymbolsnabbrev

% --------------------------------------------------
% Main content of thesis organized into chapters
% --------------------------------------------------
\mainmatter

\chapter{The basics}
\label{chap:introduction}

\section{A simple section}
\label{sec:asimplesection}

\subsection{A sub-section}
\label{ssec:asubsection}

\chapter{Figures, sub-figures and more}
\label{chap:figures}

\chapter{Let's talk tables}
\label{chap:tables}

\chapter{Equations and code}
\label{chap:equationsandcode}

% --------------------------------------------------
% End of chapters
% --------------------------------------------------
% To add a symbol (no checking of repeated symbol)
\addsymbol{$\lambda$}{wavelength}
\addsymbol{$\epsilon_r$}{relative dielectric constant}
\addsymbol{$k$}{wave number, defined as $2\pi/\lambda$}

% To add an abbreviation, which will be named at least once in the full
\addabbrev{IEEE}{Institute of Electrical and Electronics Engineers}
\addabbrev{PASS}{Phased Array System Simulator}
\addabbrev{RF}{Radio Frequency}

\backmatter

\end{document}
